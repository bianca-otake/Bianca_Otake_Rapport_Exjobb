\chapter{Theoretical Background}
\section{Radar Sensors}
In this section the radar as a sensor will be introduced, with its wide variety of applications. Invented in 1935, its development was heavily affected by World War II where the majority of its applications were for military purposes, such as detection and separation of hostile and friendly units. From the military applications it started growing into its role of a multipurpose agent able to do many things. Depending on its operating frequency, the radar is able to assist in a wide range of applications. Everything from detection of big and small obstacles, to clouds for weather forecasts when the radar operates at a lower frequency. Higher frequencies is used for detection of smaller targets, for example hand gestures.  In robotics, radars usually have an operating frequency above 24GHz where it can detect people, and larger vehicles.
\\\\
Due to its inherent ability to penetrate any weather conditions, radar has become a more important sensor when it comes to robust autonomous driving. It however suffers from different types of noise, clutter, and is susceptible to jamming. The main types of noise to expect are: internal noise from electrical components, and external noise, such  as thermal radiation. Clutter is the main source of misleading readings in terms of radar. It is a broad term to refer to when the radar signal detects targets that are not of interest (like ground, birds, sand storms, etc.), or detects echos of the signal. In comparison to the \ac{LiDAR} sensor, the radar is inferior in terms of precision and resolution, making it an obvious choice in many applications. \ac{LiDAR}, does however suffer from severe degradation in bad weather, which motivates the investigation of radar as a basis for a localization system.

% \thispagestyle{fancy}
% In this chapter, a detailed description about background of the degree project is presented together with related work. Discuss what is found useful and what is less useful. Use valid arguments. 
% \\\\
% Explain what and how prior work / prior research will be applied on or used in the degree project /work (described in this thesis). Explain why and what is not used in the degree project and give valid reasons for rejecting the work/research.
% \\\\
% Radar can also be used with other types of sensor to form a heterogenous system to perform SLAM, as in \cite{jurgens_radar-based_2020}. There radar is used for the base of the SLAM approach but the design allows integration with stereo cameras and \ac{LiDAR}s, by extracting keypoints in the same type of manner. In the industry, radar can also be used with the other types of sensors to create \ac{OGM}s that tell the user if a certain position is occupied or free, an important binary to know when planning or executing a driving route. This can however pose a problem in terms of pure localization as radar sensors, \ac{LiDAR}s, and cameras are prone to detect and see different types of sensors differently. Making an OGM recorded with a camera or \ac{LiDAR}, perhaps hard to compare with a radar OGM extracted with the same principle.   

\section{Radar Localization in Previous Work}
One of the first attempts to use radar for localization was presented in \cite{ward_vehicle_2016} where ten raw point clouds from two automotive grade 2D mid range radars were used with \ac{ICP} to retrieve a global pose measurement with the help of a pre-recorded map. A similar point-to-point ICP approach was applied in \cite{holder_real-time_2019} where multiple raw scans were batched and lonely points removed to form a local map, and extract \ac{RO} to perform \ac{SLAM}. Due to the fact that the low angular resolution and sparse nature of the radar data, raw point clouds need to be aggregated over time to enable matching between sets. 
\\\\
The same type of sensor was used in \cite{narula_automotive-radar-based_2020} where three are used to perform localization using 5 second batches of radar data. From that,  \ac{OGM}s are constructed and seen as 2D images and correlation is computed in the Fourier domain to retrieve 2D pose measurements. In \cite{yoneda_vehicle_2018} batches of radar data and OGMs are similarly used to construct 2D images that are compared, but instead of doing the correlation computations in the Fourier domain, they are done in the spatial domain, resulting in many more computations. These two OGM based methods show promising accuracy and precision and will thus form the basis of the degree project. 
\\\\
At the core of an OGM is the inverse sensor model. The inverse sensor model decides how to update the probabilities, given a radar reading. Due to the specific nature of the radar sensor \cite{slutsky_dual_2019} introduces a novel inverse sensor model, specifically designed for radar readings which results in more robust OGMs. In \cite{weston_probably_2019} a inverse sensor model for radar designed by training a deep neural network with the help of \ac{LiDAR}.
\\\\
The automotive grade radar sensor can also be used to extract key-points or so called landmarks, as done in \cite{jurgens_radar-based_2020}, and \cite{schuster_landmark_2016}. There the landmarks are used to perform SLAM. In \cite{jurgens_radar-based_2020} it is used as the foundation for a heterogeneous SLAM system able to incorporate \ac{LiDAR} and stereo cameras. The main difficulty is to extract keypoints, and perform correct correlation between sets of keypoints. This seem to provide results less accurate than the previously presented approaches in terms of mapping and localization accuracy and precision. The abundance of noise and low angular resolution experienced with short and medium range automotive grade radars make this difficult, as explained in \cite{holder_real-time_2019}. Key-point extraction methods also introduce many new tuneable parameters, decreasing the simplicity of the solution.
\\\\
Radars providing power-range spectra of the environment can also be used to perform localization, as done in \cite{cen_precise_2018}, \cite{aldera_what_2019}. In contrast to the automotive grade radar, the power range-spectrum data seem more suitable for key-point extraction and matching. In \cite{cen_precise_2018} key-point sets are matched using ICP and in \cite{aldera_what_2019} a network is trained to detect false positives.  Precise \ac{RO} is demonstrated in both studies. The type of radar used in these studies are however not common in the automotive industry due its placement and size, making it hard to apply these results to the intended scenario. 
\\\\
To conclude, there seem to be three main ways one can represent radar maps. The first is to use raw radar for matching. The second is to use the raw point clouds to construct OGMs, and the third is to use the point clouds to extract key-points from the point clouds. All implementations need to aggregate data over time to acquire a sufficient amount of information to perform matching between data sets.
% The power-range spectra contains more information and is easier to extract keypoints from as the information delivered by the sensor is less processed than that data delivered from an automotive grade sensor. 
% \\\\
% Storing the radar data using an \ac{OGM} is advantageous as it enables the user to model uncertainties. This means that "unseen regions" in the map can be modeled by an unknown probability. This is of particular interest if the vision of the radar is, for example, obstructed by a large building. 
\section{Data Association Methods}
Irrespective of what way one chooses to build the radar map one needs to decide on a method that associates scans with each other. The data association methods aim to find the translation and rotation between data sets, and thereby provide pose information. The method that is chosen will have a great impact on the final results of the localization, but its efficiency will also depend on how the radar data is represented. 
\\\\
Methods are mainly divided up in two different types: iterative, and optimal solutions. Iterative solutions are fast, and optimal solutions generally take longer time to find, but instead they guarantee the best match given the information available. The iterative solution risks converging to locally optimal solutions which is a problem when repetitive structures and radar echos are common.
\\\\
Iterative solutions have been widely and successfully implemented with \ac{LiDAR} data, to perform matching between raw scans. ICP and \ac{RANSAC} are the most common methods in point cloud, and key-point alignment. These methods work well when the resolution of the data is high, as in the \ac{LiDAR} case. But with the sparse radar data, successful matching between single scans have not yet been achieved.
\\\\
The localization is expected to perform worse in situation where RO also perform bad. In \cite{aldera_what_2019} the three main scenarios when RO experience failures are presented as  
\begin{itemize}
    \item When movement of the vehicle is non-planar 
    \item When large parts of the scene is occluded by moving or static obstacles
    \item When the surroundings experience a lack of distinct targets or structures
\end{itemize}